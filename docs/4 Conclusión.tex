\section{Conclusión}

A lo largo del proyecto, se logró diseñar e implementar una solución web full-stack funcional, segura y escalable, integrando herramientas modernas tanto en el frontend como en el backend.

Entre los principales aprendizajes que se obtuvieron durante el desarrollo de este proyecto destacan la importancia de la planificación arquitectónica (no se puede crear un buen proyecto sin antes planificarlo), la correcta separación de responsabilidades entre los distintos componentes del sistema (¿qué es trabajo del backend y qué del frontend?) y la adopción de buenas prácticas de desarrollo, como el uso de control de versiones (se usó git y GitHub, aprovechando especialmente la funcionalidad de las ramas) y contenedores.

Como áreas de mejora y expansión futura para este proyecto, se identifican:

\begin{itemize}
    \item Implementar funcionalidades avanzadas, como notificaciones automáticas de bajo inventario o integración con sistemas externos de facturación.
    \item Mejorar la interfaz de usuario para dispositivos móviles.
    \item Incorporar pruebas automatizadas.
    \item Permitir la gestión de múltiples almacenes o sucursales.
    \item Optimizar el rendimiento y la seguridad ante escenarios de mayor concurrencia.
\end{itemize}

Estas mejoras no solo enriquecerían la funcionalidad del sistema, sino que también ofrecerían una experiencia de usuario más completa y satisfactoria.

No obstante, no hay que olvidar que el sistema desarrollado cumple con los objetivos planteados: permite a los usuarios gestionar inventarios en línea, realizar operaciones fundamentales sobre los productos y exportar reportes en PDF. Además, la arquitectura modular facilita el mantenimiento y la futura expansión del sistema.
