En la actualidad, la gestión eficiente de inventarios es un reto común para empresas y organizaciones de todos los tamaños. Muchos negocios aún dependen de métodos manuales o sistemas obsoletos, lo que genera errores, falta de visibilidad y dificultades para mantener información actualizada y confiable. La mejora de estos procesos es fundamental para mejorar la organización interna y reducir errores humanos.

Este proyecto propone una solución basada en una aplicación web full-stack que permite a los usuarios gestionar sus inventarios de manera eficiente y segura. La aplicación está diseñada para facilitar el registro, consulta, actualización y eliminación de productos en tiempo real, ofreciendo una herramienta moderna y accesible para la administración de inventarios.

\section{Objetivo}

El objetivo principal de este proyecto es diseñar y desarrollar una aplicación web que permita la gestión eficiente de inventarios en línea. Esta solución busca proporcionar a los usuarios una herramienta moderna, accesible y segura para administrar los productos de una tienda o empresa, facilitando el registro, consulta, actualización y eliminación de los artículos en tiempo real. El sistema estará orientado a mejorar la organización interna, reducir errores humanos y optimizar los procesos relacionados con el manejo de inventarios.

\section{Problema por Resolver}

Actualmente, muchas organizaciones enfrentan retos significativos en la administración de sus inventarios, especialmente cuando dependen de métodos manuales o sistemas poco actualizados. Entre los principales problemas se encuentran la falta de visibilidad sobre el estado actual del inventario, la dificultad para registrar y consultar movimientos de productos (altas, bajas, modificaciones y búsquedas), la posibilidad de cometer errores en el conteo o registro, y la ausencia de reportes claros y exportables que respalden las operaciones realizadas.

\section{Propuesta de Solución}

Para abordar el problema identificado, se desarrolló una aplicación web full-stack que integra tecnologías modernas tanto en el frontend como en el backend. El frontend fue implementado con React, permitiendo una interfaz de usuario dinámica, intuitiva y visualmente atractiva. La página principal presenta de manera clara las funcionalidades disponibles y cuenta con formularios de inicio de sesión y registro para garantizar la seguridad.

El backend se implementó utilizando Java Spring Boot, creando una API REST que permite la comunicación entre el frontend y la base de datos MySQL. Esta arquitectura facilita la escalabilidad y el mantenimiento del sistema, además de asegurar la integridad y persistencia de los datos.

Una vez autenticados, los usuarios pueden realizar todas las operaciones fundamentales del inventario: agregar nuevos productos, consultar el listado y detalles de los artículos, eliminar productos existentes o modificar la cantidad y características de los mismos. Cada operación se refleja en tiempo real en la interfaz y el sistema proporciona la opción de descargar un reporte en PDF con los resultados de la operación realizada. Así, se garantiza que los usuarios cuenten con información actualizada y respaldada, mejorando la eficiencia y confiabilidad en la gestión de inventarios.
