Lorem ipsum dolor sit amet consectetur adipiscing elit. Aquí falta la introducción.

\section{Objetivo}

El objetivo principal de este proyecto es diseñar y desarrollar un sitio web que permita la gestión eficiente de inventarios en línea. Esta solución busca proporcionar a los usuarios una herramienta moderna, accesible y segura para administrar los productos de una tienda o empresa, facilitando el registro, consulta, actualización y eliminación de los artículos en tiempo real. El sistema estará orientado a mejorar la organización interna, reducir errores humanos y optimizar los procesos relacionados con el manejo de inventarios.

\section{Problema por Resolver}

En la actualidad, muchas organizaciones enfrentan retos significativos en la administración de sus inventarios, especialmente cuando dependen de métodos manuales o sistemas poco actualizados. Entre los principales problemas se encuentran la falta de visibilidad sobre el estado actual del inventario, la dificultad para registrar y consultar movimientos de productos (altas, bajas, modificaciones y búsquedas), la posibilidad de cometer errores en el conteo o registro, y la ausencia de reportes claros y exportables que respalden las operaciones realizadas.

El problema asignado consiste en la necesidad de contar con una solución tecnológica que permita realizar inventarios en línea, de manera que los usuarios puedan acceder al sistema desde cualquier lugar, registrar nuevos productos, consultar el listado y detalles de los artículos, eliminar productos obsoletos o modificar las cantidades y características existentes. Además, se requiere que el sistema ofrezca la posibilidad de visualizar los resultados de las operaciones realizadas y exportar dicha información en formato PDF, lo que facilitará la documentación y el respaldo de los movimientos efectuados en el inventario.

\section{Propuesta de Solución}

Para abordar el problema identificado, se propone el desarrollo de una aplicación web full-stack que integre tecnologías modernas tanto en el frontend como en el backend. El frontend estará desarrollado con React, lo que permitirá crear una interfaz de usuario dinámica, intuitiva y visualmente atractiva. La página principal del sitio presentará de manera clara las funcionalidades disponibles y contará con formularios de inicio de sesión y registro para garantizar la seguridad.

El backend se implementará utilizando Java Spring Boot, encargado de gestionar la lógica de negocio y exponer una API REST que permita la comunicación entre el frontend y la base de datos MySQL. Esta arquitectura facilitará la escalabilidad y el mantenimiento del sistema, además de asegurar la integridad y persistencia de los datos.

Una vez autenticados, los usuarios podrán realizar todas las operaciones fundamentales del inventario: agregar nuevos productos, consultar el listado y detalles de los artículos, eliminar productos existentes o modificar la cantidad y características de los mismos. Cada operación será reflejada en tiempo real en la interfaz y el sistema proporcionará la opción de descargar un reporte en PDF con los resultados de la operación realizada. De esta manera, se garantiza que los usuarios cuenten con información actualizada y respaldada, mejorando la eficiencia y confiabilidad en la gestión de inventarios.
